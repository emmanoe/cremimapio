%%%%%%%%%%%%%%pre-ambule%%%%%%%%%%%%%%%%%%%%
\documentclass{article}
\usepackage{fancyhdr}
\pagestyle{fancy}
\lhead{}
\chead{}
\rhead{}
\lfoot{Programmation système}
\cfoot{Groupe 6}
\rfoot{\thepage}
\renewcommand{\headrulewidth}{0.4pt}
\renewcommand{\footrulewidth}{0.4pt}
\usepackage{graphicx}
\usepackage[french]{babel}
\title{\textbf{Rapport du projet de programmation système}}
\author{Aza, Willem, Tremor Sullyvan, Delar Emmanoe}
%%%%%%%%%%%%%%%%%%%%%%%%%%%%%%%%%%%%%%%%%%%%%

\begin{document}
\maketitle
\thispagestyle{fancy}

%%%%%%%%%%%%%%%%Logo université%%%%%%%%%%%%%%%
\begin{figure}[!b]
		\centering
		\includegraphics[height=4cm]{logo.jpg}
	\end{figure}
%%%%%%%%%%%%%%%%%%%%%%%%%%%%%%%%%%%%%%%%%%%%%%

\newpage
\tableofcontents

\newpage
\section{Introduction}
En petit groupe de 3 \'etudiants, nous avons r\'ealis\'e un projet qui faisait appel au cours de programation syst\`eme que nous avons suivi tout au long du semestre. Le but de ce projet \'etait de d\'evelopper quelques m\'ecanisme de base destin\'es \`a servir dans le d\'eveloppement d'un petit jeu de plateforme 2D. Le projet s'est d\'eroul\'e en plusieurs parties distinctes. La premi\`ere partie portait sur la mise en place d’un m\'ecanisme de sauvegarde et de chargement de la carte utilis\'ee dans le jeu. La deuxième partie portait sur la gestion des temporisateurs permettant de planifier les diff\'erents \'ev\'enements du jeu.

\section{Manipulation de fichiers}
petite intro
	\subsection{map-save}
	decrire map save
	\subsection{map-load}
	decrire map load
	\subsection{Utilitaire de manipulation de carte}
	decrire maputil


\section{Gestion des temporisateurs}
A continuer

\end{document}